\section{Rešavanje naloge}\label{sec:jedro}
%
Nalogo bomo rešimo z \emph{Lagrange}ovimi enačbami 2. reda;
dinamično ravnotežje sistema je opisano z enačbo:
\begin{equation}\label{eq:Lagrange2}
  \frac{\textrm{d}}{\textrm{d}t}
  \frac{\partial T}{\partial \dot{q}_j}-
  \frac{\partial T}{\partial q_j}
  =
  Q_j\qquad
  \textrm{za vse $j$.}
\end{equation}
Kjer je $T$ \emph{kinetična energija sistema}, $Q$ pa
\emph{posplošena sila}. Sistem ima $P$=2 prostostnih stopenj in
$N$=2 nevezanih posplošenih koordinat:
\begin{eqnarray}\label{eq:PospKoord}
  q_1=\varphi\\
  q_2=\psi
\end{eqnarray}

\emph{Kinetična energija sistema} je:
\begin{equation}\label{eq:T}
  T=
  \frac{1}{2}\,J_O\,\dot{\varphi}^2+
  \frac{1}{2}\,m_2\,v_B^2,
\end{equation}
kjer je $J_O$ \emph{masni vztrajnostni moment} valja okoli vrtišča
O in $v_B$ \emph{hitrost translacije masne točke} B.
\begin{equation}\label{eq:JO}
  J_O=
  \frac{m_1\,R^2}{2}
\end{equation}
Da določimo hitrost točke B določimo najprej njeno lego:
\begin{equation}\label{eq:xB}
  x_B=
  (l_0-R\,\varphi+R\,\psi)\,
  \sin(\psi)+
  R\,\cos(\psi)
\end{equation}
\begin{equation}\label{eq:yB}
  y_B=
  (l_0-R\,\varphi+R\,\psi)\,
  \cos(\psi)-
  R\,\sin(\psi)
\end{equation}
Z odvajanjem krajevnih koordinat dobimo hitrosti:
\begin{equation}\label{eq:dxB}
  \dot{x}_B=
  +(l_0-R\,\varphi+R\,\psi)\,
  \dot{\psi}\,\cos(\psi)-
  R\,\dot{\varphi}\,\sin(\psi)
\end{equation}
\begin{equation}\label{eq:dyB}
  \dot{y}_B=
  -(l_0-R\,\varphi+R\,\psi)\,
  \dot{\psi}\,\sin(\psi)-
  R\,\dot{\varphi}\,\cos(\psi)
\end{equation}
Sledi:
\begin{equation}\label{eq:vB2}
  v_B^2=\dot{x}_B^2+\dot{y}_B^2=
  (l_0-R\,\varphi+R\,\psi)^2\,
  \dot{\psi}^2
  +R^2\,\dot{\varphi}^2
\end{equation}
Sedaj vstavimo enačbi~(\ref{eq:JO}) in~(\ref{eq:vB2}) v
enačbo~(\ref{eq:T}) in izpeljemo
\begin{equation}\label{eq:Tizpelj}
  T=
  \frac{1}{4}\,
  (m_1+2\,m_2)\,
  R^2\,\dot{\varphi}^2+
  \frac{1}{2}\,m_2\,
  (l_0-R\,\varphi+R\,\psi)^2\,\dot{\psi}^2
\end{equation}
Kinetično energijo sedaj parcialno odvajamo kakor narekuje
ravnotežna enačba~(\ref{eq:Lagrange2}), najprej za posplošeno
koordinato $\varphi$:
\begin{eqnarray}\label{eq:dPosp1}
  \frac{\partial T}{\partial \dot{\varphi}}&=&
  \frac{1}{2}\,(m_1+2\,m_2)\,R^2\,\dot{\varphi}\\
%
  \frac{\textrm{d}}{\textrm{d}t}
  \frac{\partial T}{\partial \dot{\varphi}}&=&
  \frac{1}{2}\,(m_1+2\,m_2)\,R^2\,\ddot{\varphi}\\
%
  \frac{\partial T}{\partial \varphi}&=&
  -m_2\,R\,(l_0-R\,\varphi+R\,\psi)\,\dot{\psi}^2\\
\end{eqnarray}
Podobno izpeljemo za posplošeno koordinato $\psi$:
\begin{eqnarray}\label{eq:dPosp2}
  \frac{\partial T}{\partial \dot{\psi}}&=&
  m_2\,(l_0-R\,\varphi+R\,\psi)^2\,\dot{\psi}\\
%
  \frac{\textrm{d}}{\textrm{d}t}
  \frac{\partial T}{\partial \dot{\psi}}&=&
  m_2\,(l_0-R\,\varphi+R\,\psi)^2\,\ddot{\psi}+
  2\,m_2\,R\,(l_0-R\,\varphi+R\,\psi)\,(\dot{\psi}-\dot{\varphi})\dot{\psi}\\
%
  \frac{\partial T}{\partial \psi}&=&
  m_2\,R\,(l_0-R\,\varphi+R\,\psi)\,\dot{\psi}^2\\
\end{eqnarray}

Glede na ravnotežno enačbo~(\ref{eq:Lagrange2}) moramo določiti še
posplošeno silo, ki jo izpeljemo iz izraza za \emph{virtualno
delo}:
\begin{equation}\label{eq:W}
  \delta W=
   Q_{\varphi}\,\delta \varphi
  +Q_{\psi}\,\delta \psi=
   M\,\delta \varphi
   m_2\,g\,\delta y_B,
\end{equation}
kjer variacijo $y_B$ moramo izračunati:
\begin{equation}\label{eq:varYB}
  \delta y_B=
              \frac{\partial y_B}{\partial \varphi} \,\delta \varphi
              +\frac{\partial y_B}{\partial \psi} \,\delta \psi
            =
            -R\,\cos(\psi)\,\delta \psi
            -(l_0-R\,\varphi+R\,\psi)\,\sin(\psi)\,\delta \psi
\end{equation}
Za virtualno delo torej izpeljemo izraz:
\begin{equation}\label{eq:WIzpel}
  \delta W=
    (M-m_2\,g\,R\,\cos(\psi))\,\delta \varphi
   +(-m_2\,g\,(l_0-R\,\varphi+R\,\psi)\,\sin(\psi))\,\delta \psi
\end{equation}
Sledi, da sta posplošeni sili:
\begin{eqnarray}\label{eq:Q}
  Q_{\varphi} &=&M-m_2\,g\,R\,\cos(\psi)\\
  Q_{\psi}&=&-m_2\,g\,(l_0-R\,\varphi+R\,\psi)\,\sin(\psi)
\end{eqnarray}
Èe sedaj enačbo za dinamično ravnotežje~(\ref{eq:Lagrange2})
zapišemo za vsako posplošeno koordinato posebej:
\begin{eqnarray}\label{eq:Lagrange2Komp1}
  \frac{\textrm{d}}{\textrm{d}t}
  \frac{\partial T}{\partial \dot{\varphi}_j}-
  \frac{\partial T}{\partial \varphi_j}
  =
  Q_{\varphi}\\\label{eq:Lagrange2Komp2}
  \frac{\textrm{d}}{\textrm{d}t}
  \frac{\partial T}{\partial \dot{\psi}_j}-
  \frac{\partial T}{\partial \psi_j}
  =
  Q_{\psi}
\end{eqnarray}
Z vstavljanjem zgoraj izpeljanih izrazov za kinetično energijo in
posplošeno silo v enačbi~(\ref{eq:Lagrange2Komp1})
in~(\ref{eq:Lagrange2Komp2}) dobimo sistem dveh diferencialnih
enačb gibanja:
\begin{eqnarray}\label{eq:resitev}
   (l_0-R\,\varphi+R\,\psi)\,\ddot{\psi}
  +R(\dot{\psi}^2+2\,\dot{\varphi}\,\dot{\psi})
  &=&
  -g\, \sin(\psi)\\
   \frac{1}{2}(m_1+2\,m_2)\,R\,\ddot{\varphi}
  +m_2(l_0-R\,\varphi+R\,\psi)\,\dot{\psi}^2
  &=&
  M-m_2\,R\,g\,\cos(\psi)
\end{eqnarray}
